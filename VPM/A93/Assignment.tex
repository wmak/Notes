\documentclass{article}
\usepackage[utf8]{inputenc}
\usepackage{hyperref}

\linespread{2}

\usepackage[margin=0.5in]{geometry}

\title{Listening to Music Assignment}
\author{William Mak}
\date{November 2014}

\begin{document}

\maketitle

% Introduction Paragraph.
For this assignment I chose to compare two different performances of the jazz 
song {\it Take Five} I decided to use one of Dave Brubeck's performances of the 
song heard here on youtube:
\url{https://www.youtube.com/watch?v=vmDDOFXSgAs}
as well as Sachel studios take on take five as heard here on youtube:
\url{https://www.youtube.com/watch?v=GLF46JKkCNg}
they are immediately very contrasting performances of the same piece. One being
a Pakistani Orchestra playing, and the other a jazz quartet's performance. The
piece was originally composed by Paul Desmond one of the original performers in
Dave Brubeck's quartet. Soon after it's initial release in 1959 the song became
a one hit wonder and the best selling jazz single of all time.

% Time
Listening to Sachel Studios interpretation of the piece and examining it's tempo
one can hear that it's tempo is quite fast, this is especially evident with the
man's performance on the tabla, the percussive instrument that makes the beat of
the song so promininent. Listening to the original composition the Drum set in
Dave Brubeck's performance starts the song a bit slower than the Sachel
performance. Both though have the same meter.

% Pitch
Though the melody of the songs is the same. Their pitch is quite different
especially since Sachel Studios interpretation uses a Sitar and a orchestra of
violins for the melody. Which is of a much higher pitch than the saxaphone that
Dave Brubeck plays. The melody of the two have the same Range and Motion but
because of their slightly different rhythms the shape and form are quite
distinct. As well their textures are quite distinct from one another. The Sachel
Studio version has a much more polyphonous texture having multiple melodic lines
showing off various instruments. Especially with the counterpoint between the
Sitar and the Orchestra of strings. Meanwhile Brubeck's version is more of a
Homophony with the saxaphone taking the main melodic line with the drums doing
accompaniment.
\end{document}
