\documentclass{article}
\usepackage[utf8]{inputenc}
\usepackage{hyperref}

\linespread{2}

\usepackage[margin=1in]{geometry}

\title{Listening to Music Assignment}
\author{William Mak}
\date{November 2014}

\begin{document}

\maketitle

% Introduction Paragraph.
For this assignment I chose to compare two different performances of the jazz 
song {\it Take Five} I decided to use one of Dave Brubeck's performances of the 
song heard here on youtube:
\url{https://www.youtube.com/watch?v=vmDDOFXSgAs}
as well as Sachel studios take on take five as heard here on youtube:
\url{https://www.youtube.com/watch?v=GLF46JKkCNg}
they are immediately very contrasting performances of the same piece. One being
a Pakistani Orchestra playing, and the other a jazz quartet's performance. The
piece was originally composed by Paul Desmond one of the original performers in
Dave Brubeck's quartet. Soon after it's initial release in 1959 the song became
a one hit wonder and the best selling jazz single of all time. Sachel Jazz
orchestra's rendition of Take Five was so imaginative that Brubeck himself is
quoted to have describe it as "the most interesting and different recording of
Take Five that I have ever heard.". The two performances are distinct especially
in their chosen instuments and style, but still maintain the melody and tune
that make Take Five so very iconic.

\\% Time
Listening to Sachel Studios interpretation of the piece and examining it's tempo
one can hear that it's tempo is quite fast, this is especially evident with the
man's performance on the tabla, the percussive instrument that makes the beat of
the song so promininent. Listening to the original composition the Drum set in
Dave Brubeck's performance starts the song a bit slower than the Sachel
performance. Both though have the same meter, especially since the name of the
song comes from the meter used 5/4. At the time of composition this was
something especially of note since most songs of that time came in 4/4 time or
3/4 waltz time. The drums from both performances are what really push the tempo.
The meter that they convey is very easy to hear thanks to them. But the tabla
has a much stronger dynamic. Starting louder than any of the other instruments
and then decrescendoing as the piece plays out. Meanwhile the drumkit in
Brubeck's version started as an accompaniment to the saxaphone and Crescendoing
to a solo of it's own.

\\% Pitch
Though the melody of the songs is the same. Their pitch is quite different
especially since Sachel Studios interpretation uses a Sitar and a orchestra of
violins for the melody. Which is of a much higher pitch than the saxaphone that
Dave Brubeck plays. The melody of the two have the same Range and Motion but
because of their slightly different rhythms the shape and form are quite
distinct. As well their textures are quite distinct from one another. The Sachel
Studio version has a much more polyphonous texture having multiple melodic lines
showing off various instruments. Especially with the counterpoint between the
Sitar and the Orchestra of strings. Meanwhile Brubeck's version is more of a
Homophony with the saxaphone taking the main melodic line with the drums doing
accompaniment. 

\\% Structure
The structure of the two pieces are quite different though. Especially since the
instruments used are so different. In the Sachel take of the song there is a
large amount of contrast. The song is in strophic form going from the Sitar and
the Violins playing the initial tune. And then to a second tune with the guitar
and tabla taking the stage and continuing this. Until finally at the end all the
instruments join in for the big finale. Meanwhile Brubeck's rendition of the
song had more of a Ternary form, starting with the saxaphone backed with the
drums and piano. The song then turns to the drums taking priority, and as the
song concludes it returns to the melody that the saxaphone played originally. 

\\% Timbre
As I've mentioned a few times in this essay the timbre of the two songs are very
different. The general pitch of the Sachel Studio cover was higher and smoother
since the classification of their musical instruments was distinctly different.
While Brubecks quartet was composed of a saxaphone, a piano, a drumkit and a
bass. The Sachel symphony has a Tabla, a Sitar, Violins, several Chellos and a
Guitar. These two sets of instruments have very different styles of
articulation. While the Brubeck's cover was more detached and accented the
Sachel studio cover was much smoother especially because of their use of
violins. The range of the Sachel studio cover and the original Brubeck recording
were quite similar as well. They were both about middlei Tenor range with some 
short periods where it would get higher into the Soprano range. The dynamics 
play into this as well as each melodic line comes in and out in Sachel's version
they bring a different style of sound. Most notably the Sitar and Guitar rarely
play together but work very well against each other. Or when the Guitar and
tabla are playing one right after the other.

\\% Context
Both recordings have very different styles and genres. While the melody remains
the same between pieces There genres are quite different. Brubeck's piece was
much more jazzy especially with the drumkit and saxaphone. But the Sachel
performance was much more orchestral especially having such a large variety of
instruments. As well given their types of instruments the Sachel studio cover
was very culturally different. I don't think there would be that many Tablas or
Sitars in western music. As well watching the video they seem to have some
pretty high production values, having recording studios and sucha wide
assortment of isntruments. The same is true for the recording of take five given
that we know that it was the top selling jazz single of all time. As well the
technique of both sets of performers are exceptional. Watching the man play the
tabla was particularly astounding.

\\% Conclusion
Both pieces are wonderfully performed and create a completely different
experience for the listenener. They were both created to push what people expect
from music. For brubeck 5/4 was a time signature rarely used by anyone and his
use of it was called revolutionary by some. Meanwhile Sachel studios took a jazz
standard and played it in a style and genre that nobody was expecting them to
use. Brubeck created a hit single that is famous even today after 50 years and
his song has been played by many different musicians from The Simpsons to a
Pakistani Orchestra.
\end{document}
