\documentclass{article}
\usepackage[utf8]{inputenc}
\usepackage{geometry}
\usepackage{mathtools}
\usepackage{listings}
\lstset{language=Python} %declare python as language
\DeclarePairedDelimiter{\ceil}{\lceil}{\rceil}

\setcounter{totalnumber}{100}

\title{L03 \- Search}
\author{William Mak}
\date{January 05 2015}

\begin{document}

\maketitle
\section{Search}
\begin{enumerate}
	\item We need a way to determine which nodes are reachable from current
		node.
	\item commonly called a successor() function
	\item goal test.
	\begin{enumerate}
		\item Check graph for loops
		\item we don't go into any of these loops.
		\item therefore we have a tree.
	\end{enumerate}
	\item Our search algorithm then is straight forward
	\begin{enumerate}
		\item \textbf{select a node} to expand, we'll keep track of a list of
			active nodes.
		\begin{enumerate}
			\item DFS
				\begin{enumerate}
					\item \textbf{implemented:} FIFO Queue
					\item \textbf{runtime:} $O(b^s)$
				\end{enumerate}
			\item BFS
				\begin{enumerate}
					\item \textbf{implemented:} Stack + Recursion
					\item \textbf{runtime:} $O(b^m)$
				\end{enumerate}
			\item UCS \- Uniform Cost Search.
				\begin{enumerate}
					\item expand cheapest node first. A less special BFS.
					\item \textbf{implemented:} Priority Queue + Heap.
				\end{enumerate}
		\end{enumerate}
		\item Check for solution.
		\item \textbf{Add} children of newly expanded node to \underline{active
			node list} in some \textbf{specified order}.
	\end{enumerate}
\end{enumerate}

\section{Heuristic Search}
\begin{enumerate}
	\item Get an estimate for how close some node is from goal.
	\item Manhattan Distance, Euclidian Distance
	\item come up with a good heuristic \textbf{(Good Exam Question)}
	\item A* Search
	\begin{enumerate}
		\item Heuristic Search
		\item Compute 'cost' of a node as $f(n) = g(n) + h(n)$
		\item $g(n) = $ Cost to get to n from s
		\item $h(n) = $ Heurstic estimate to goal.
	\end{enumerate}
	\item Properties of an admissible heuristic: $h(n) \leq h^*(n)$ where $h^*(n)$
		is the true cost to get to n from S
\end{enumerate}

\end{document}
