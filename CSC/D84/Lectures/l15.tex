\documentclass{article}
\usepackage[utf8]{inputenc}
\usepackage{geometry}
\usepackage{mathtools}
\usepackage{listings}
\usepackage{tikz}
\usetikzlibrary{arrows}
\lstset{language=Python} %declare python as language
\DeclarePairedDelimiter{\ceil}{\lceil}{\rceil}

\setcounter{totalnumber}{100}

\title{L15}
\author{William Mak}

\begin{document}

\maketitle
\section{Probabilistic Reasoning}
\begin{enumeration}
	\item Keep track of most likely states
	\item Random variable: variable whose value depends on a random event
		\subitem discrete: takes on a discrete values, eg dice roll: 1...6
		\subitem continuous: PDF function intgegrates to 1
	\item for n variables, domain size d table has $d^n$ entries
		can't learn joint distribution so we have to be clever
	\item Events (start of reasoning with probability)
		\subitem What if only know values for a subset of random variables
	\item Conditional Distribution p(a|b) = p(a,b)/p(b)
	\item Bayes Rule.
\end{enumeration}

\end{document}
