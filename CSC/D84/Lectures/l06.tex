\documentclass{article}
\usepackage[utf8]{inputenc}
\usepackage{geometry}
\usepackage{mathtools}
\usepackage{listings}
\usepackage{tikz}
\usetikzlibrary{arrows}
\lstset{language=Python} %declare python as language
\DeclarePairedDelimiter{\ceil}{\lceil}{\rceil}

\setcounter{totalnumber}{100}

\title{L06 \- Faster Constraint Satisfaction}
\author{William Mak}
\date{January 05 2015}

\begin{document}

\maketitle
\section{running faster}
How to make Constraint Satisfaction graph faster? (Midterm q)\\
\begin{enumerate}
	\item Variable with fewest possible values left to explore. This is because
		this cuts down on the number of subtree we will have to explore later
		on. The fewer subtrees the less work we'll have to do.
	\item Pick variable that intervenes in the most active constraints. Do this
		if there's a tie in the first principle.
	\item But what about choosing the value of the variable?
		\begin{enumerate}
			\item Choose value that leaves most choices for neighbors. This
				increases chance that things will work.
		\end{enumerate}
\end{enumerate}

\subsection{Early search termination}
Arc Consistency checking\\
an arc $x \to y$ is consistent if for every possible vlaue in X there is at
least one in y that does not break constraints.

\subsection{Running time}
\begin{enumerate}
	\item General CSP, n variables, domain of size d
		\subitem $O(d^n)$
	\item Split problem into n/c subproblems of size C
		\subitem $O(\frac{n}{c} \times d^c)$
	\item If CSP has tree structure (No loops)
		\subitem $O(nd^2)$
	\item Suppose I chose C variables whose value I fix 
		\subitem $O(d^c \times (n-c) \times d^2)$
\end{enumerate}

\end{document}
