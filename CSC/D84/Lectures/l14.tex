\documentclass{article}
\usepackage[utf8]{inputenc}
\usepackage{geometry}
\usepackage{mathtools}
\usepackage{listings}
\usepackage{tikz}
\usetikzlibrary{arrows}
\lstset{language=Python} %declare python as language
\DeclarePairedDelimiter{\ceil}{\lceil}{\rceil}

\setcounter{totalnumber}{100}

\title{L14}
\author{William Mak}

\begin{document}

\maketitle
\section{Q - learning}
\begin{enumeration}
	\item what are we learning in Q-learning ????
	\item Knowledge about which action yields greater reward [for each state]
		\subitem does not generalize and it can be huge
	\item Generalizing R.L -> feature based learning
	\item Ieda, replace state configuration with feature vector
		\subitem ie, $S_i = [f_{1i}, f_{2i}, f_{3i}, ...]$
	\item No more Q table
		\subitem directly eval $Q(s_i) = w_1f_{1i} + w_2f_{2i} + ... w_kf_{ki}$
		\subitem $ = \Sigma_{j=1}^{k} w_jf_{ji}$
		\subitem $Q(s, a) = r + \gamma max(Q(S')) - Q(s)$
		\subitem update: $w_i = w_i + \alpha[difference] f_i(s)$
		\subitem stop when weight updates are tiny
		\subitem AKA Gradient
\end{enumeration}

\end{document}
