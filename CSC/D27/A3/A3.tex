\documentclass{article}
\usepackage[utf8]{inputenc}
\usepackage{geometry}
\usepackage{mathtools}
\usepackage{enumerate}
\usepackage{listings}
\DeclarePairedDelimiter{\ceil}{\lceil}{\rceil}

\geometry{margin=0.5in}

\title{Assignment 2}
\author{William Mak}
\date{November 07 2014}

\begin{document}
\maketitle

\section{Course Feedback}
See Email

\section{So, You Want to be a Hacker?}
\begin{enumerate}[A]
	\item
		If the marker had not turned off the default stack-protector and the the
		user enters a string over certain length then they would observe an
		error stating:
		\begin{verbatim}
		*** stack smashing detected ***: ./marker terminated
		======= Backtrace: =========
		\end{verbatim}\\
		etc...
	\item
		The string values I used were:
		\begin{verbatim}
		Name: aaaaaaaaaaaaaaaaaaaaaaaaaaaaaaaaaaaaaaaaiz
		Mark: hi
		Rubric: hellooooooooo
		\end{verbatim}\\
		The important part of the Name variable are the last two characters.
		This is because the $34^{th}$ character causes an overflow to occur. An
		overflow occurs because the way auth and name are stored causes auth to 
		be 60 bytes away so we can just set it to 'iz'
	\item
		No. In this particular case it would have been protected. Because the
		user would write in the opposite direction from auth. But if 
		'char name[34]` came before `int auth=0` then it would still be 
		possible.
\end{enumerate}

\section{Abuse of Set-UID File Permissions}
Given 60 seconds of access to another student's unattended keyboard I would go
into their "\$HOME/.ssh/authorized\_keys" file and add my own ssh key. This way
later on I would be able to ssh into their computer without needing their
password.

\end{document}
