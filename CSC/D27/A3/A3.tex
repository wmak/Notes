\documentclass{article}
\usepackage[utf8]{inputenc}
\usepackage{geometry}
\usepackage{mathtools}
\usepackage{enumerate}
\usepackage{listings}
\DeclarePairedDelimiter{\ceil}{\lceil}{\rceil}

\geometry{margin=0.5in}

\title{Assignment 2}
\author{William Mak}
\date{November 07 2014}

\begin{document}
\maketitle

\section{Course Feedback}
See Email

\section{So, You Want to be a Hacker?}
\begin{enumerate}[A]
	\item
		If the marker had not turned off the default stack-protector and the the
		user enters a string over certain length then they would observe an
		error stating:
		\begin{verbatim}
		*** stack smashing detected ***: ./marker terminated
		======= Backtrace: =========
		\end{verbatim}\\
		etc...
	\item
		The string values I used were:
		\begin{verbatim}
		Name: aaaaaaaaaaaaaaaaaaaaaaaaaaaaaaaaaaaaaaaaiz
		Mark: hi
		Rubric: hellooooooooo
		\end{verbatim}\\
		The important part of the Name variable are the last two characters.
		This is because the $34^{th}$ character causes an overflow to occur. An
		overflow occurs because the way auth and name are stored causes auth to 
		be 60 bytes away so we can just set it to 'iz'
	\item
		No. In this particular case it would have been protected. Because the
		user would write in the opposite direction from auth. But if 
		'char name[34]` came before `int auth=0` then it would still be 
		possible.
\end{enumerate}

\section{Web Security}
See attached files for answers to A, D, E

\section{Wifi Router Vulnerabilities}
\begin{enumerate}[A]
	\item
		For this attack as an attacker I would first use db-ip.com to find all
		possible Canadian addresses if I knew about this vulnerability with
		Bell. Or if the screenshot is from his router then I would know that I
		can go on 76.65.189.43. 
		I would then use netcat to send the following request.
		request(http://192.168.2.1/login?u=admin&p=admin). The ones that do I
		would then send a new DNS server. With my new DNS server I would have
		the google ip redirect to my kim.dot.com.
	\item
		Sitting outside of the professor's condo the attacker would be able to
		send TCP packets to the router with a different source address. And
		because the attacker would be right next to the router their address
		would remain unchanged. Now the attacker can send the same requests as
		before and modify the DNS 
	\item
		A hacker could very easily inject some malicious javascript into
		youvyou.com that would authenticate with the administrative account. And
		then aftewards modify the DNS server yet again to set google to be
		kim.dot.com
\end{enumerate}

\section{Denial of Service}
\begin{enumerate}[A]
	\item
		For this attack I would send a TCP packet into the network-connection
		with the reset flag set to one. This way the other computer will think
		that the connection is no longer working.
	\item 
		It's well known that TCP does not have a builtin defense against
		spoofing of TCP resets(RST). For example in 2007 after Comcast decided
		to do this attack on their users a RFC was released detailing the
		failings of rfc \url{https://tools.ietf.org/html/rfc4953}
	\item
		Yes provided that they can monitor the network and determine things such
		as the sequence number. And that they can spoof their IP. If they cannot
		do either one of these things then it would not be possible for them to
		conduct a similar account
	\item
		No Subscribers will not be able to protect themselves even if they use
		SSL encrypted packets. This is because the attackers can target either
		client or server.
		% http://arxiv.org/pdf/1208.2357.pdf
\end{enumerate}

\end{document}
