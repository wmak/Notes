\documentclass{article}
\usepackage[utf8]{inputenc}
\usepackage{geometry}
\usepackage{mathtools}
\usepackage{enumerate}
\usepackage{listings}
\DeclarePairedDelimiter{\ceil}{\lceil}{\rceil}

\geometry{margin=0.5in}

\title{Assignment 2}
\author{William Mak}
\date{November 07 2014}

\begin{document}
\maketitle

\section{Course Feedback}
See Email
\section{Key-Handling Vulnerabilities with Public-Key Systems}
\begin{enumerate}[A]
\item
    \begin{enumerate}[i]
		\item Eve will not know enough information to decipher any future
			messages between Alice and Bob
        \item No this scheme is insecure against Mallory she would be able to
			modify $g^a mod p$ or $g^b mod p$ with her own values. Alice and Bob
			will still be able to send messages between each other but Mallory
			would also be able to read them.
    \end{enumerate}
\item
    \begin{enumerate}[i]
        \item Yes this will be secure against Eve, even though she can see the
			public keys this is not enough for her to decipher any of the
			messages.
        \item No Mallory could insert her own public key instead of either or
			both Alice and Bob's keys. In this case her own private key would be
			able to decipher any future messages.
    \end{enumerate}
\item
    \begin{enumerate}[i]
        \item Yes, there's absolutely nothing Eve can do to read the messages
			between Alice and Bob.
        \item Yes, Even if Mallory knew their public keys she wouldn't be able
			to read the messages or create her own.
    \end{enumerate}
\item
    \begin{enumerate}[i]
        \item Yes, there's absolutely nothing Eve can do to read the messages
			between Alice and Bob.
        \item Yes, Mallory would not be able to do anything to the messages
			betwwen Alice and Bob.
        \item Yes, In the very unlikely case that Mallory can brute force either
			of their private keys in the previous case she would be able to
			modify messages between Alice and Bob. But in this case she would
			have less than a day to brute force the key and she would only be
			able to modify messages for one day.
    \end{enumerate}
\end{enumerate}
\section{Don't be Evil}
See Mathlab submission
\section{Exploiting Weak RSA Keys to Decrypt HTTPS Packets}
Primes:\\
$p =
15543364719846102734035905265021007874527909072857707530407929290687101099950235380124854710034333308285660\\202980757696503925683873746491877865507719899027251207102262278680571817169261263452242836721127096300509656928\\177929591710741117876178047883330343700515792044255540432173197$\\
$q = 8194124624414046878093826113$

\newpage
\lstinputlisting{hax.html}
\section{SSL Stripping}
\begin{enumerate}[A]
	\item SSLstrip relies on severla use cases. First it assumes that a user will
		not type out the full "https://" in their browser since it needs to
		pretend to be a HTTPS session while being HTTP. Secondly it requires
		there to be traffic from non-ssl pages to secure ssl pages. 
	\item The best way to thrwart someone using SSLStrip is to ensure their site
		has SSL enabled on every page.
	\item A site like facebook's best way to protect against this is to make it
		extremely clear when switching to and from SSL so the user has no
		confusion as to which is which. As well as informing the user of the
		risks if this happens.
	\item hi %TODO(william)%
\end{enumerate}
\section{ARP-Cache Poisoning}
\begin{enumerate}[A]
	\item A simplistic ARP attack would be to send an ARP reply to a victim
		telling them the false location to something system critical, for
		example the router. This would cause the victims system to be unable to
		access that resource. This would only work if the victim's system is
		accepting unsolicited replies.
	\item If Mallory wanted to perform a Man in the Middle attack on Alice and
		Bob she would start by sending a "reply" to Bob stating that her
		computer's MAC address was Alice's $a$, now Bob will think that
		Mallory's IP address $M$ is Alices $A$. Next Mallory sends a "reply" to
		Alice stating that her MAC address was Bob's $b$, now Bob will think
		Mallory's IP address $M$ is Bob's $B$. Now finally whenever
		Alice sends a packet to Bob her packet will be received by Mallory
		instead who can do whatever she likes. And the forwards it along to Bob.
		and vice versa for any traffic coming from Bob.
	\item The easiest solution would be to use static IP tables this means that
		there is no confusion as to which IP address is who on the network. This
		would address any issues with preexisting computers on the network. But
		this would still be problematic if when a new computer joins the network
		there is already a Man in the Middle.
	\item This tool would only slow an attacker down, in both scenarios
		described in previous questions they only have at most 3 different IP
		addresses and MAC addresses. Which means they would be only able to
		launch this attack against one or two devices in the network, every 30
		seconds. Which with a patient attacker isn't that big a deal.
\end{enumerate}
\section{TCP SYN-Cookies }
\begin{enumerate}[A]
	\item Yes, if the attacker responds with an invalid {\bf ACK} code. As well
		the attacker can spoof their IP address in the SYN which means that the
		server will send their requests to a falsified IP address and then
		waste resources on these invalid cookies.
	\item No, probably not since most of the IP addresses in the SYN will be
		valid the server will have no trouble getting a response from them. But
		in the case that the attacker has enough bandwidth to overpower the
		server it would still go down from the overwhelming number of requests.
\end{enumerate}

\end{document}
