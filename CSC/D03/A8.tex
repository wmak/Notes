\documentclass{article}
\usepackage[utf8]{inputenc}

\linespread{2}

\usepackage[margin=0.5in]{geometry}

\title{D03 What are the limits of copyright?}
\author{William Mak}
\date{November 2014}

\begin{document}

\maketitle

\section{Eglinton Software}
York Mills Systems has violated Eglinton Software's rights under Scarberian
copyright laws. Their application SuperPrax has the exact same features that
Doc's Office has. Knowing that our software has a monopoly on the market they
have created a piece of software that is identical to Doc's Office with a few
minor changes. According to section (c) in section 34.1 the output of an
algorithm, in this case that of Doc's Office graphical user interface. Is a
protected intellectual property in which copyright inheres. The design of this
interface was newly created by Eglinton Software and is not a prevailing
standard as seen by York Mills Systems previous graphical interface which was
already usable.\\

As well according to section (b) in section 34.1 the expression of Doc's Office
is clearly protected inttelectual property. The expression of code that creates
the graphical interface is newly created by Eglinton Software. As well it is not
obvious especially since York Mills Systems had created their own graphical
interface before deciding to steal the one created by Eglinton Software. So 
clearly section (d) of section 34.1
does not and should not apply to SuperPrax. As well there does not and should
not exist a prevailing standard for a graphical interface especially the use of
our menus, keyboard commands and shortcuts. According to all this it should be
clear to anyone with a relatively simple grasp of the law would agree that
SuperPrax is clearly violating copyright law. Their actions here are clearly to
copy Doc's Office graphical interface and they are obviously doing so because
they know they have a far inferior product in comparison to York Mills Systems.

\section{York Mills Systems}
Eglinton Software claims that we are violating copyright law. They're claim is
that SuperPrax is copying Doc's Office. According to section (a) of section 34.1
Any algorithms, theorems and mathematical facts are not copyrightable. Which
means that any of those that SuperPrax and Doc's Office might share by chance
is not an issue in this case. As well the expression of SuperPrax's code for the
various algorithms and graphical interfaces are claerly distinct. These sections
also do not apply since according to section (d) in section 34.1 the graphical
interface is not under copyright protection since they are obvious to anyone who
is skilled in the art of designing a user experience. Especially since this was
clearly the only interface that would do well with medical staff. As well
section (e) states since Doc's Office has a monopoly and every doctor uses their
software their graphical design has become the prevailing standard.\\

Eglinton has a monopoly on the market and are clearly afraid to innovate to face
new competitors. Their suing of York Mills Systems is clear proof of this. They
know that if they stop other companies from using a similar graphical interface
they will be able to maintain their monopoly. As well our products are clearly
distinct given the extra features that SuperPrax has developed to bring added
features to the users. 
\section{Software professional}
Personally I would support the Scarberian League of Programming Freedom. With
something as trivial as a graphical interface it's illogical to limit
development and innovation in such a way. SuperPrax and Doc's Office need each
other to push for better software. If Doc's Office maintains it monopoly then
they have no reason to improve their functionality or their interface. This is
pretty obvious with how SuperPrax so easily was able to add so many features to
the text editor, but was unable to join the market because of the interface
which shouldn't be too relevant as long as it's still usable by doctors.
\end{document}
