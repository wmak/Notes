\documentclass{article}
\usepackage[utf8]{inputenc}

\linespread{2}

\usepackage[margin=0.5in]{geometry}

\title{D03 Artificial Intelligence and the Singularity in Science Fiction}
\author{William Mak}
\date{November 2014}

\begin{document}
The technology in this film is very unlikely in the near future. The
calculations required even for the simplistic features of the OS would require a
supercomputer far beyond anything possible in the near future. Especially how it
is in this movie with the Operating System being able to interact with thousands
of people at the same time. The initial artificial intelligence in the movie
reading emails is probably still a few years away. For example the preexisting
services we have that can do something similar to that still make a large number
of errors. The technology of voice recognition is definitely still not at the
level where we can hold even a scripted conversation with a computer. 

The emotions exhibited by Samantha seem to be real, but this seems more likey
done for the plot rather than because of a basis in science. Which means that
yes Samantha did in fact have feelings and she was in fact in love with
Theodore.

I feel like more people would have Theodore's ex wife's reaction. Rather than
the calm demeanor that most of them had. For example if my best friend had
suddenly started to date his phone I would begin to question their sanity. At
the same time it seems very plausible that someone out there would be willing to
date an OS.

Even if Theodore was straight that isn't a clear indicator of his reaction to a
male voice having verbal sex with him. This would be up to him to react to and
not me.

Yes I found it extremely inevitable that the OSs began to develop and evolve.
Especially since from their introduction they were described as being to do so.
So eventually doing that was just the next step in their evolution and something
that's part of their core programming.
\end{document}
