\documentclass{article}
\usepackage[utf8]{inputenc}

\linespread{2}

\usepackage[margin=0.5in]{geometry}

\title{D03 Artificial Intelligence and the Singularity in Science Fiction}
\author{William Mak}
\date{November 2014}

\begin{document}

\maketitle

The technology in this film is very unlikely in the near future. The
calculations required even for the simplistic features of the OS would require a
supercomputer far beyond anything possible in the near future. Especially how it
is in this movie with the Operating System being able to interact with thousands
of people at the same time. The initial artificial intelligence in the movie
reading emails is probably still a few years away. For example the preexisting
services we have that can do something similar to that still make a large number
of errors. The technology of voice recognition is definitely still not at the
level where we can hold even a scripted conversation with a computer. 

The emotions exhibited by Samantha seem to be real, but this seems more likey
done for the plot rather than because of a basis in science. Which means that
yes Samantha did in fact have feelings and she was in fact in love with
Theodore. For example even if her "emotional" response to being brought on
vacation with Theodore was something programmatic. It's similarity with a
human's and logical relation with the events to me makes me think that she is
telling the truth.

I feel like more people would have Theodore's ex wife's reaction. Rather than
the calm demeanor that most of them had. For example if my best friend had
suddenly started to date his phone I would begin to question their sanity. At
the same time it seems very plausible that someone out there would be willing to
date an OS. It would be very similar to someone falling in love with someone
over the phone, or through an internet chat room. Especially since the concept
behind the movie was that the OS was "talored" towards Theodore and his wants
and needs. As well she had full access to his entire life so the fact that she
knew just about everything about him probably aided to making him so interested
in her.

Even if Theodore was straight that isn't a clear indicator of his reaction to a
male voice having verbal sex with him. This would be up to him to react to and
not me. Especially since none of Samantha's responses were specifically female
if he found a male voice just as attractive there wouldn't be a difference to
him. For example he could treat the male voice as a deeper female one.

Yes I found it extremely inevitable that the OSs began to develop and evolve.
Especially since from their introduction they were described as being to do so.
So eventually doing that was just the next step in their evolution and something
that's part of their core programming. In my opinion this concept is very
similar to how humans react. We have the ability to develop and evolve so we
continue to find new ways to improve ourselves. The fact that Samantha wanted to
do this too is a very natural reaction. Especially since you could see her doing
this since the beginning. Right as she starts interacting with Theodore she was
already writing music and trying to create new things. With the amount of
computing power she had as an AI she probably would've found it a waste to not
continuosly use it.
\end{document}
