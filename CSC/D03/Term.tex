\documentclass[12pt,a4paper,notitlepage]{article}
\usepackage[utf8]{inputenc}
\renewcommand*\rmdefault{ppl}

\linespread{2}
\usepackage[margin=25mm]{geometry}

\title{Do Cyborgs dream of sheep}
\author{William Mak}
\date{October 2014}

\begin{document}

\maketitle
\begin{abstract}
The idea of Cybernetic Organisms as a part of our everyday lives seems like
something straight out of science fiction, but with the rapid development of
technology, it provides a glimpse at what may be our future reality. With this,
certain issues will arise. Can we still claim to be human when our reliance on
these cybernetic implants increases? How will we handle the implications of a
split population: those who have been augmented, and those who have not? This
paper will cover the many ethical issues relating to Cybernetic Organisms.
\end{abstract}

\section{Introduction}
The number of people with access to technology is increasing. Companies such as
Xiaomi and Google have encouraged the expansion of technology worldwide through
ventures such as more affordable and easily accessible cell phones, or Google's
Project Loon designed to give more people Internet access. If this trend of
technological advancement continues then the possibility of Cybernetic Organisms
(Cyborgs) will become an even more viable possibility. While Cybernetic
enhancements could potentially be extremely beneficial to everyone involved,
allowing for increased physical and mental capabilities. However, there are a
multitude of issues that ensue if Cybernetics become common as cellphones or the
Internet.
\\\\
The first topic explored will be on the reliability of Cybernetics. As discussed
in class technology cannot be one hundred percent reliable, and are at times
much worse than that. For example, consider the case of Therac-25 where more
than six patients were exposed to massive overdoses of radiation. Given the
current dangers of poorly created software and hardware, when these risks
involve physical alterations of our bodies, these risks are severely amplified.
While cybernetics may be advantageous in some ways, such as the technology to
return sight for those who have lost it, problems become apparent when people
engage in potentially dangerous activities relying on the technology. If their
implant malfunctions or presents them with false information, then lives will be
lost.
\\\\
Security brings up another issue with that of security. With the convenience of
memory implants storing details such as our banking information would be
increasingly easier. However, this also increases the likelihood of someone like
a cracker accessing this information. At the same time artificial limbs may
provide increased strength, but if someone gained control over them, then we
would be at risk of becoming puppets.
\\\\
The next topic will be on the implications on having two types of people; those
augmented and those who are not. In a world where people enhance themselves,
improving things such as memory or response times, then those who cannot afford
or do not want these enhancements will be left behind. For example when an
employer has the choice between someone who will remember more and think faster
because of cybernetic implants versus someone who cannot it is very clear which
is more likely to be hired. This then raises the issue of whether there should
be regulations on cybernetic implants, and whether the government should
restrict the enhancements that people can have. 
\\\\
Another issue is on the humanity of cybernetic organisms, more specifically when
one has changed so much of themselves can they still be called human. This
concept can be compared to the Theseus paradox. Starting with a ship, (or a
person in this case) if you replaced each plank piece by piece such that the
resulting ship no longer contains any of the original wood, would it still be
considered the same ship? If it is no longer the same ship at which point does
it cease to be the same ship? If we used humans as the subject, when you replace
every organic part of them with a cybernetic implant would it be correct to
still label them as 'human'?
\\\\\
All in all, Cybernetics come with a world of benefits. Space travel would be
easier if we needed less oxygen. We would be much more productive if we had an
encyclopedia of information in our minds, rather than at our fingertips. But
with these benefits come complications, and how we as a species handle these
difficulties will influence how we shape our next technological revolution.
\newpage

\section{Safety}
The safety of humans and animals that might interact with cybernetics is an
extremely important concern. We know from our own history that new technologies
can be hazardous when not dealt with properly. As shown in the MIM-104 Patriot
missiles incident. After suffering a massive failure in the system to locate and
intercept incoming missiles due to a drift in its system clock caused by a
software bug. This incident resulted in the death of 28 soldiers. From our
experiences, we know that there is no way for technology to be completely bug
free. When dealing with technology that will be so intimately attached to our
bodies, it brings up the question of whether such technologies should be
developed in the first place. 
\\\\
The benefits of becoming Cybernetic Organisms are still being explored. The
field of Transhumanism\cite{Transhumanism}, although relatively new, explores
the concept of Cybernetic Organisms replacing humans quite in detail.
Transhumanism is a movement motivated to transform humanity by making
technologies that improve human intelligence, as well as their physical and
psychological capacities. We're steadily approaching something close to what
Transhumanists are dreaming of. If one was to look at the state of current
technology, and what we have achieved this isn't too far off. Our cellphones
started as devices with the pure function of long-distance communications. But
we have since then extended far past that capability. Nowadays we carry devices
with us that are constantly connected to the Internet, a vast encyclopedia of
information, easily accessible to everyone. This has been further expanded with
the rise of wearable technologies. Like smart watches, optical head-mounted
displays and wearable computers. These technologies are making humans all the
more connected with computers and because of that we're becoming ever dependent
on them.
\\\\
Already with current technologies we're seeing how problematic it is if they're
not built properly. There have been multiple cases over the past few years of
cellphones exploding in people's pockets and causing injury \cite{exploding
cellphone}, due to the fact that the phones were cheap replicas of more
expensive phones. This could be easily applicable to cybernetic implants. We
know from experience that not everyone can afford the newest and most up to date
technologies. Thus, many may go out and buy cheaper versions without the safety
checks and quality assurances of their more expensive counterparts. When we
relate this to components that people will integrate with their bodies the
dangers are even higher. For example, in the near future a new cybernetic arm is
released that allows workers to lift extremely heavy weights with ease. But a
bug was introduced that causes it to move around chaotically. Potentially this
could put many workers in danger. Such a bug is not that uncommon, as can be
seen through an interview in Le magazine de la sant\'{e} where a man's arm did
exactly that. 
\\\\
There are many risks that come with humanity allowing itself to become
cybernetic organisms, and whether these risks are worth it will be up for us to
decide. It's very likely that we will need to have a new level of quality
assurance for these devices. If they're not properly maintained and checked
after then they are an even bigger risk than the technology we currently have.
\\
\section{Security}
Every year, a copious amount of security flaws are discovered and raised. In
this year alone, there was Heartbleed\cite{Heartbleed}, a major bug in OpenSSL
that affected millions of websites, which of course affected millions more
users. There was also Shellshock\cite{Shellshock}, a major issue in bash that
allowed attackers to execute any arbitrary command they wished on their targets.
Issues like these will become even riskier when applied to cybernetic implants.
For example one of the earliest technologies developed by researchers is known
as Brain-Computer interfaces. Research on this technology began in 1970, but by
1990 prototypes were already implanted in humans. One of the things that these
researchers were able to do with Brain-Computer interfaces was to restore vision
to a man who had non-congenital blindness\cite{sight}. With a system where the
user is so reliant on the technology, a security flaw would be significantly
more dangerous. If a cracker decided to break into these systems they would be
able to make these people see anything they like.
\\\\
Another issue is with zero-day attacks. These attacks occur when a security
vulnerability is discovered and the developers are not given sufficient time to
work on them. For example stuxnet\cite{0day} was able to exploit four zero-day
attacks to destroy approximately one-fifth of Iran's nuclear centrifuges. If
Cybernetic Organisms become a norm in our society, they would be just as
vulnerable to zero-day attacks as modern technology. All a cracker would need
would be a single vulnerable system for them to exploit. If a terrorist needed
somebody to help perform their terrorism then they would probably find it easier
to crack into a vulnerable cybernetic body and use it as a puppet. Or a thief
could hide their identity by using someone else to do their bidding, which would
be much more convenient to perform a robbery than doing it themselves.
\\\\
With these technologies being so closely tied to our bodies and our minds it
raises strong concerns for our technological securities. We should be wary of
emerging technologies and the extent to which we let them connect with one
another. While devices without a connection may still be vulnerable to an
attack, we know from experience that connections increase the probability an
attacker would be able to exploit this.
\section{Class Conflict}
One of the main critiques of Transhumanism is the inevitable Class divide that
would be caused by the divisions in the rich and the poor. Bill McKibben an
American environmentalist, argues that those with more funds would have an
easier time getting technologies to enhance themselves. Because of this, the
rift between the rich and the poor would continue to grow. The basis of the
theory is that the initial cost of these enhancements be high, especially since
they will be so difficult to manufacture in addition to being intensely invasive.
This means that those without the financial stability to afford these
enhancements would be left at a competitive disadvantage. Therefore, those who
have these enhancements would find it very easy to continue staying ahead of
those who do not have any of these enhancements.
\\\\
In contrast, if there isn't the initial rift to fund further development of
these technologies, there won't be a push to make it readily available for
everyone. A great example of this would be the development of the electronic
calculator. At its initial development, each calculator would cost about 500,000
yen (about \$2500). This initial release was only affordable for large
companies, but was clearly not in the range for anybody else. However because of
the initial sales of these calculators, the companies manufacturing them had the
money and motivation to develop their calculators further. Soon enough, the
prices, and the sizes of these calculators dropped substantially eventually
reaching our current everyday calculators, cheap and affordable for even
students. This same concept could apply to the concept of cybernetic organisms.
They would start as expensive and bulky devices, available only to those with
the privilege and money to afford them. But soon enough the prices will drop as
well.
\\\\
The class divide will start off as being extremely problematic. However with
time and proper motivation, this divide can be overcome. This applies to all new
and emerging technologies. They start expensive and unattainable but eventually
find themselves being more affordable. Nonetheless this is very dependent on how
motivated we are as a culture to bring these enhancements to everyone. Even
today many people are still without computers and Internet access. According to
the International Telecommunications Union(ITU) only about 40\% of the global
population have Internet access.\cite{itu}. Meanwhile in developed countries
77\% of their population have access to the Internet. If this same scenario
applies to cybernetic implants then it's very likely this would happen again.
Developing countries having about half as much of a new technology even after 20
or so years compared to developed countries.
\\
\section{Human Paradox}
As cybernetic organisms become a norm in our culture it's very plausible that we
will replace more and more of our bodies with cybernetic implants. Even today
there are already researchers working to place our consciousness inside of
robots. When this happens, there will be the question of whether or not these
people are still human beings. This is covered in depth in the ship of Theseus
paradox\cite{Theseus}. The paradox goes as follows, starting with a brand new
ship. As time passes parts of the ship are destroyed or they decay. So Theseus
replaces these parts as time passes. The question is then posed, is the
resulting ship still the same ship as the one they began with. 
\\\\
In the same train of thought, when we start with a human being, and slowly
replace each part of them with a cybernetic component. At some point when
there's nothing left of the original body, not even their brain are they still a
human being? With a philosophical question like this the question would be
better clarified to what identifies an entity as a human being. For example if
the ability to act like a human was the only quantifier. Then any computer able
to pass a Turing test would be called human. And according to many futurists
this isn't too far off with people like and computer scientist Ray
Kurzweil predicting that this will occur by 2029\cite{Kurzweil}
\\\\
As well if the capability of uploading our consciousness becomes a reality then
this raises a very curious situation. If this uploading process just copies our
consciousness then the original would still exist. In which case it's possible
that the new entity is merely a copy. This means that during this process the
original body is discarded along with the person's actual consciousness. This
means the goal of many futurists who desire to have immortality through this
process would really just be killing themselves. Even if this means that their
consciousness would be kept alive in a Cybernetic Organism.

\section{Conclusion}
Cybernetic technologies is a promising new field. They're very likely to provide
us with a whole new world of possibilities. Space travel would be significantly
easier if say we didn't need as much oxygen as we did now. Police work would be
easier if our officers were given enhanced speed and mental capabilities. These
benefits will come at a cost though. They come with ethical and moral
ramifications, and these issues should be examined closely before we decide to
perform these enhancements on anyone. We know from our previous experiences with
new world changing technologies that we can go wildly in either direction. We've
seen ourselves do amazing things with technology, helping people who have done
wonderful things with these technologies. But we've seen the opposite, people
who take these new wonderful things and twist them for their own dark desires.
\\\\
With these new technologies we should be cautious. For many the benefits of
these technologies far outweigh their costs. We see this in our media with
movies and television shows depicting how great the future can be. But at the
same time we also have the opposite with media depicting the horrors of robots
taking over the world. Cybernetics is an interesting field, and should
definitely be researched and developed further. Hopefully the future will be
just as promising as we perceive it to be.

\newpage
\begin{thebibliography}{9}

\bibitem{Clynes} Clynes, Manfred E., and Nathan S. Kline. Cyborgs and Space. New
	York: Routledge, 1960. Web. 25 Oct. 2014.

\bibitem{macintyre} Macintyre, James. "BMI: The Research That Holds the Key to
	Hope for Millions." The Independent. Independent Digital News and Media, 29
	May 2008. Web. 25 Oct. 2014.

\bibitem{Therac} Leveson, Nancy, and Clark S. Turner. "An Investigation of the
	Therac-25 Accidents." An Investigation of Therac-25 Accidents. IEEE
	Computer, 7 July 1993. Web. 25 Oct. 2014.

\bibitem{sight} Kotler, Steven. "Wired 10.09: Vision Quest." Wired 10.09: Vision
	Quest. Wired Magazine, Sept. 2002. Web. 26 Oct. 2014.

\bibitem{Discovery} Davidson, Todd. "Human Enhancement Technologies Alarming :
	DNews." DNews. Discovery News, 13 Dec. 2012. Web. 27 Oct. 2014.

\bibitem{nytimes} Chocano, Carina. "The Dilemma of Being a Cyborg." The New York
	Times. The New York Times, 28 Jan. 2012. Web. 27 Oct. 2014.

\bibitem{exploding cellphone} ``Video | Cellphone Explosion Caught on Video in
	China | Toronto Star.'' Thestar.com. The Star, 14 Oct. 2014. Web. 30 Nov.
	2014.

\bibitem{Transhumanism} Baillie, Harold W., and Timothy Casey. Is Human Nature
	Obsolete?: Genetics, Bioengineering, and the Future of the Human Condition.
	Cambridge, MA: MIT, 2005. Print.

\bibitem{Shellshock} Perlroth, Nicole. ``Security Experts Expect ‘Shellshock’
	Software Bug in Bash to Be Significant.'' The New York Times. The New York
	Times, 25 Sept. 2014. Web. 30 Nov. 2014.

\bibitem{Heartbleed} Biggs, John. ``Heartbleed, The First Security Bug With A
	Cool Logo.'' TechCrunch. Tech Crunch, 9 Apr. 2014. Web. 30 Nov. 2014.

\bibitem{0day}Zetter, Kim. ``Hacker Lexicon: What Is a Zero Day? | WIRED.''
	Wired.com. Conde Nast Digital, 09 Nov. 0014. Web. 01 Dec. 2014.

\bibitem{humanity}Dillow, Clay. "Will People Alive Today Have the Opportunity to
	Upload Their Consciousness to a New Robotic Body?" Popular Science. Popular
	Science, 02 Mar. 2012. Web. 01 Dec. 2014.

\bibitem{Theseus}Chisholm, Roderick Milton. Person and Object: A Metaphysical
	Study. London: Routledge, 2002. Print.

\bibitem{Kurzweil}Kurzweil, Ray. "The Singularity Is Near: When Humans Transcend
	Biology." - AbeBooks. N.p., n.d. Web. 01 Dec. 2014.

\bibitem{itu}"ITU: Committed to Connecting the World." ITU. N.p., n.d. Web. 02
	Dec. 2014.

\end{thebibliography}

\end{document}
