\documentclass[12pt,a4paper,notitlepage]{article}
\usepackage[utf8]{inputenc}

\linespread{2}
\usepackage[margin=25mm]{geometry}

\title{Do Cyborgs dream of sheep}
\author{William Mak}
\date{October 2014}

\begin{document}

\maketitle
    \begin{abstract}
        The concept of Cybernetic Organisms becoming common place is something out of science fiction, but as technology progresses this will eventually become a reality. With this comes some issues. Can we rely on our cybernetic implants, the implications of having two types of people; those augmented and those who are not, and can we still call ourselves human after we would have changed so much.
    \end{abstract}

\section{Introduction}
    The number of people with access to technology is increasing, thanks to 
companies like Google with their project Loon more people are able to get onto
the internet. Or with companies like Xiaomi which seek to provide cheaper
affordable cellphones for everyone. If this trend of technological advancement 
continues then the possibility of Cybernetic Organisms or Cyborgs will become an
even more viable possibility. While Cybernetic enhancements could potentially be
extremely beneficial to everyone involved, giving them increased physical and
mental capabilities. There are a multitude of issues that will exist and already
exist if Cybernetics become just as common as cellphones or the internet.
\\\\
    The first topic explored here will be the reliability of Cybernetics. As
we've discussed in class technology cannot be one hundred percent reliable, and 
are at times much worse than that. An easy example of this would be Therac-25
where in atleast six patients were involved with massive overdoses of radiation.
Given the dangers with badly created software and hardware that already exist
today. When these issues are applied to something that is so closely tied with
our bodies they're severely amplified. For example there already exists
technology to return sight for those who have lost it. The issue then arises 
where these people are now doing something potentially dangerous which is
dependent on their sight. If their implant malfunctions or presents them with 
false information then lives will be lost.
\\\\
    Another issue is that of Security with the convenience of something like a
memory implant storing things such as our banking information would be very
convenient. But this also means that someone like a cracker would be very likely
to try to get into our implants. At the same time with things like artificial
limbs to give increased strength if someone could control these then we would be
at risk at becoming someones puppet.
\\\\
    The next topic will be on the implications on having two types of people;
those augmented and those who are not. In a world where people will begin to
enhance themselves improving things such as memory or response times those who
cannot afford or do not want these enhancements will be left behind. For
example when an employer has the choice between someone who can remember more 
and think faster because of cybernetic implants versus someone who cannot it is
very clear who is more likely to be hired. This then raises the issue of whether
there should be regulations on cybernetic implants, whether the government
should decide to restrict people on what enhancements they can have. 
\\\\
    Another issue is on the humanity of cybernetic implants, more specifically
when one has changed so much of themselves can they stil be called human. This
concept can be compared to the theseus paradox. Starting with a ship, or a
person in this case. If you replaced each plank piece by piece such that the
resulting ship no longer contains any of the original wood, whether it still be
considered the same ship. If it is no longer the same ship at which point does
it cease to be the same ship. In this case if you start with a person and
replace every organic part of them with a cybernetic implant would one still
call them human.
\\\\\
    All in all Cybernetics come with a world of benefits. Space travel would be
easier if we needed less oxygen, we would be much more productive if we had an
encyclopedia of information in our minds rather than at our fingertips. But
these benefits come with complications and how we as a species handle these
complications will influence how we shape our next technological revolution.
    
\section{Safety}

\section{Security}
\section{Class Conflict}
\section{Human Paradox}
\section{Conclusion}

\begin{thebibliography}{9}

\bibitem{Clynes}
Clynes, Manfred E., and Nathan S. Kline. Cyborgs and Space. New York: Routledge, 1960. Web. 25 Oct. 2014.

\bibitem{macintyre}
Macintyre, James. "BMI: The Research That Holds the Key to Hope for Millions." The Independent. Independent Digital News and Media, 29 May 2008. Web. 25 Oct. 2014.

\bibitem{Therac}
Leveson, Nancy, and Clark S. Turner. "An Investigation of the Therac-25 Accidents." An Investigation of Therac-25 Accidents. IEEE Computer, 7 July 1993. Web. 25 Oct. 2014.

\bibitem{sight}
Kotler, Steven. "Wired 10.09: Vision Quest." Wired 10.09: Vision Quest. Wired Magazine, Sept. 2002. Web. 26 Oct. 2014.

\bibitem{Discovery}
Davidson, Todd. "Human Enhancement Technologies Alarming : DNews." DNews. Discovery News, 13 Dec. 2012. Web. 27 Oct. 2014.

\bibitem{nytimes}
Chocano, Carina. "The Dilemma of Being a Cyborg." The New York Times. The New York Times, 28 Jan. 2012. Web. 27 Oct. 2014.

\end{thebibliography}

\end{document}
