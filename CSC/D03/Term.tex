\documentclass[12pt,a4paper,notitlepage]{article}
\usepackage[utf8]{inputenc}

\linespread{2}
\usepackage[margin=25mm]{geometry}

\title{Do Cyborgs dream of sheep}
\author{William Mak}
\date{October 2014}

\begin{document}

\maketitle
\begin{abstract}
	The concept of Cybernetic Organisms becoming common place is something out
of science fiction, but as technology progresses this will eventually become a 
reality. With this comes some issues. Can we rely on our cybernetic implants,
the implications of having two types of people; those augmented and those who
are not, and can we still call ourselves human after we would have changed so 
much.
\end{abstract}

\section{Introduction}
    The number of people with access to technology is increasing, thanks to 
companies like Google with their project Loon more people are able to get onto
the Internet. Or with companies like Xiaomi which seek to provide cheaper
affordable cellphones for everyone. If this trend of technological advancement 
continues then the possibility of Cybernetic Organisms or Cyborgs will become an
even more viable possibility. While Cybernetic enhancements could potentially be
extremely beneficial to everyone involved, giving them increased physical and
mental capabilities. There are a multitude of issues that will exist and already
exist if Cybernetics become just as common as cellphones or the Internet.
\\\\
    The first topic explored here will be the reliability of Cybernetics. As
we've discussed in class technology cannot be one hundred percent reliable, and 
are at times much worse than that. An easy example of this would be Therac-25
where in at least six patients were involved with massive overdoses of radiation.
Given the dangers with badly created software and hardware that already exist
today. When these issues are applied to something that is so closely tied with
our bodies they're severely amplified. For example there already exists
technology to return sight for those who have lost it. The issue then arises 
where these people are now doing something potentially dangerous which is
dependent on their sight. If their implant malfunctions or presents them with 
false information then lives will be lost.
\\\\
    Another issue is that of Security with the convenience of something like a
memory implant storing things such as our banking information would be very
convenient. But this also means that someone like a cracker would be very likely
to try to get into our implants. At the same time with things like artificial
limbs to give increased strength if someone could control these then we would be
at risk at becoming someones puppet.
\\\\
    The next topic will be on the implications on having two types of people;
those augmented and those who are not. In a world where people will began to
enhance themselves improving things such as memory or response times those who
cannot afford or do not want these enhancements will be left behind. For
example when an employer has the choice between someone who can remember more 
and think faster because of cybernetic implants versus someone who cannot it is
very clear who is more likely to be hired. This then raises the issue of whether
there should be regulations on cybernetic implants, whether the government
should decide to restrict people on what enhancements they can have. 
\\\\
    Another issue is on the humanity of cybernetic implants, more specifically
when one has changed so much of themselves can they still be called human. This
concept can be compared to the Theseus paradox. Starting with a ship, or a
person in this case. If you replaced each plank piece by piece such that the
resulting ship no longer contains any of the original wood, whether it still be
considered the same ship. If it is no longer the same ship at which point does
it cease to be the same ship. In this case if you start with a person and
replace every organic part of them with a cybernetic implant would one still
call them human.
\\\\\
    All in all Cybernetics come with a world of benefits. Space travel would be
easier if we needed less oxygen, we would be much more productive if we had an
encyclopedia of information in our minds rather than at our fingertips. But
these benefits come with complications and how we as a species handle these
complications will influence how we shape our next technological revolution.
\newpage

\section{Safety}
The safety of humans and animals that might interact with cybernetics is an
extremely important issue. We know from our own history with new technology that
they can be hazardous when not dealt with properly. For example from class we
know about the MIM-104 Patriot missiles, that because of a software bug causing
a drift in it's system clock. This drift caused a failure in the system to
locate and intercept incoming missiles. From our experience with technology we
know that there is no way for any of them to be bug free. So especially with a
technology that will so closely be attached to our bodies whether or not this
technology should be developed is a very difficult question.
\\\\
The benefits of becoming Cybernetic Organisms are still being explored. For
example the field of Transhumanism\cite{Transhumanism} though relatively new
explores this concept of Cybernetic Organisms replacing humans quite in depth.
Transhumanism is a movement motivated to transforming humanity, by making
technologies that improve human intelligence, as well as their physical and
psychological capacities. We're steadily approaching something close to what
these Transhumanists are dreaming of. If one was to look at the state of
technology that we have with us this isn't too far off. Our cellphones started
as devices that allowed us to talk with people far away.  But have since then
extended far past that capability. Nowadays we carry devices with us that are
constantly connected to the Internet which can be a vast encyclopedia of
information, constantly at the fingertips of everyone.  And this has even
expanded past that we're seeing the rise of wearable technologies. With things
like smart watches, optical head-mounted displays and wearable computers. These
technologies are making humans all the more connected with computers and because
of that we're becoming ever dependent on them.
\\\\
And already with these current technologies we're seeing how problematic it is
if they're not built properly. For example there are multiple cases over the
past few years of people's cellphones exploding in their pockets \cite{exploding
cellphone}. With these cases the issue was that the person who was injured used
cheap replicas of more expensive phones. Which in the case of cybernetic
implants is entirely a possibility. We know from experience that not everyone
can afford the newest and most up to date technologies. But they still want
them, so people go out and buy cheaper versions. Without nearly as much safety
checks and quality assurances as their more expensive counterparts. When this is
done with components that people will integrate with their bodies the dangers
are even higher. For example say a new cybernetic arm was released that allowed
workers to lift extremely heavy weights with ease. But a bug was introduced that
caused it to move around chaotically. Potentially this could put many workers in
danger. And a bug like this isn't that uncommon, Le magazine de la sant\'{e} had
an interview with a man whose arm did exactly that. 
\\\\
There are many risks that come with allowing humanity to become cybernetic
organisms. And whether these risks are worth it will be up for us to decide.
It's very likely that we will need to have a new level of quality assurance for
these devices. If they're not properly maintained and checked after then they
are an even bigger risk than current the technology we currently have.
\\
\section{Security}
Every year a copious amount of security flaws are discovered and raised. This
year alone we've seen Heartbleed\cite{Heartbleed} a major bug in OpenSSL that
affected millions of websites. Which of course affected millions more users.
There was also Shellshock\cite{Shellshock} a major issue in bash that allowed
attackers to execute any command they wished on their targets. These issues are
even riskier when applied to cybernetic implants. For example one of the
earliest technologies developed by researchers is known as Brain-Computer
interfaces. Research on this technology began in 1970 but by 1990 prototypes
have already been implanted in humans. One of the things that these researchers
were able to do was to restore vision to a man who had non-congenital
blindness\cite{sight}. With a system like this where the user is so reliant on
the technology a security flaw would be magnitudes more dangerous. If a cracker
decided to break into these systems they would be able to make these people see
anything they like.
\\\\
Another issue is with zero-day attacks. These attacks occur when a security
vulnerability is discovered and the developers are not given any time to work on
them. For example stuxnet\cite{0day} was able to exploit four zero-day attacks
to destroy around one-fifth of Iran's nuclear centrifuges. And this is just one
of many cases, so when Cybernetic Organisms become a norm in our society they
are just as vulnerable to zero-day attacks as current technology. All a cracker
would need is a single vulnerable system and they would be able to exploit it.
If a terrorist needed somebody to help perform their terrorism then they would
probably find it easier to crack into a vulnerable cybernetic body and then use
it like a puppet. Or if a thief could use someone else to do their bidding,
which would be much more convenient to perform a robbery than doing it
themselves.
\\
\section{Class Conflict}
One of the main critiques of Transhumanism is the Class divide that would be
caused by the divisions in the rich and the poor. For example Bill McKibben an
American environmentalist argues that those with more funds would have an easier
time getting technologies that would enhance themselves. Then because of this
the rift between the rich and the poor would continue to grow. The idea here is
that the initial cost of these enhancements will have to be high. Especially
since they're going to be so difficult to manufacture as well as being so
invasive. Which then means without the financial stability to afford this would
be left without the ability to compete with those who do. Therefore those who
have these enhancements would find it very easy to continue staying ahead of
these who do not have any of these enhancements.
\\\\
But on the contrary to this. If there isn't the initial rift to fund further
development of these technologies there will not be a push to make it available
for everyone.  A great example of this was the development of the electronic
calculator. At it's initial development each calculator would cost about 500,000
yen which is about \$2500. This initial release was affordable for large
companies but was clearly not in the range for anybody else. But because of the
sales of these initial calculators these companies had the money and the
motivation to develop their calculators further. Soon enough the prices, and the
sizes of these calculators dropped substantially until we have the current state
of calculators. Cheap and affordable for even a student. This same concept
applies to the concept of cybernetic organisms. They would start as expensive
and bulky devices, affordable only to those with the privilege and money to
afford them.  But soon enough thanks to those people the prices will drop as
well.
\\\\
The class divide will start of as being extremely problematic. But with enough
time and with the proper motivations this divide could be overcome. This is and
will be true for all new and emerging technologies. They began as expensive and
eventually find themselves being more affordable. All of this though is very
dependent on how motivated we are as a culture to bring these enhancements to
everyone. Cause even today many people are still without computers and Internet
access. According to the ITU, the International Telecommunications Union only
about 40\% of the global population have Internet access.\cite{itu}. Meanwhile
the developed world has about 77\% of it's populations with Internet Access. If
this pattern would continue with cybernetic implants then it's very likely this
would happen again. Developing countries having about half as much of a new
technology even after 20 or so years compared to developed countries.
\\
\section{Human Paradox}
As cybernetic organisms become a norm in our culture it's very plausible that we
will replace more and more of our bodies with cybernetic implants. Even today
their are already researchers hard at work to place our consciousness into
robots. When this happens the question will arise whether or not these people
are still human beings. This is very well phrased in the ship of Theseus
paradox\cite{Theseus}. The paradox goes as follows, starting with a brand new
ship. As time passes parts of the ship are destroyed or they decay. So Theseus
replaces these parts as time passes. The question is then posed, is the
resulting ship still the same ship as the one they began with. 
\\\\
In the same train of thought, when we start with a human being, and slowly
replace each part of them with a cybernetic component. At some point when
there's nothing left of the original body, not even their brain are they still a
human being? With a philosophical question like this the question would be
better clarified to what identifies an entity as a human being. For example if
the ability to act like a human was the only quantifier. Then any computer able
to pass a Turing test would be called human. And according to many futurists
this isn't too far off with people like futurist and computer scientist Ray
Kurzweil predicting that this will occur by 2029\cite{Kurzweil}
\\\\
As well if the capability of uploading our consciousness becomes a reality then
this raises a very curious situation. If this uploading process just copies our
consciousness then the original would still exist. In which case it's possible
that the new entity is merely a copy. This means that during this process the
original body is discarded along with the person's actual consciousness. This
means the goal of many futurists who desire to have immortality through this
process would really just be killing themselves. Even if this means that their
consciousness would be kept alive in a Cybernetic Organism.

\section{Conclusion}


\newpage
\begin{thebibliography}{9}

\bibitem{Clynes} Clynes, Manfred E., and Nathan S. Kline. Cyborgs and Space. New
	York: Routledge, 1960. Web. 25 Oct. 2014.

\bibitem{macintyre} Macintyre, James. "BMI: The Research That Holds the Key to
	Hope for Millions." The Independent. Independent Digital News and Media, 29
	May 2008. Web. 25 Oct. 2014.

\bibitem{Therac} Leveson, Nancy, and Clark S. Turner. "An Investigation of the
	Therac-25 Accidents." An Investigation of Therac-25 Accidents. IEEE
	Computer, 7 July 1993. Web. 25 Oct. 2014.

\bibitem{sight} Kotler, Steven. "Wired 10.09: Vision Quest." Wired 10.09: Vision
	Quest. Wired Magazine, Sept. 2002. Web. 26 Oct. 2014.

\bibitem{Discovery} Davidson, Todd. "Human Enhancement Technologies Alarming :
	DNews." DNews. Discovery News, 13 Dec. 2012. Web. 27 Oct. 2014.

\bibitem{nytimes} Chocano, Carina. "The Dilemma of Being a Cyborg." The New York
	Times. The New York Times, 28 Jan. 2012. Web. 27 Oct. 2014.

\bibitem{exploding cellphone} ``Video | Cellphone Explosion Caught on Video in
	China | Toronto Star.'' Thestar.com. The Star, 14 Oct. 2014. Web. 30 Nov.
	2014.

\bibitem{Transhumanism} Baillie, Harold W., and Timothy Casey. Is Human Nature
	Obsolete?: Genetics, Bioengineering, and the Future of the Human Condition.
	Cambridge, MA: MIT, 2005.  Print.

\bibitem{Shellshock} Perlroth, Nicole. ``Security Experts Expect ‘Shellshock’
	Software Bug in Bash to Be Significant.'' The New York Times. The New York
	Times, 25 Sept. 2014. Web. 30 Nov. 2014.

\bibitem{Heartbleed} Biggs, John. ``Heartbleed, The First Security Bug With A
	Cool Logo.'' TechCrunch. Tech Crunch, 9 Apr. 2014. Web. 30 Nov. 2014.

\bibitem{0day}Zetter, Kim. ``Hacker Lexicon: What Is a Zero Day? | WIRED.''
	Wired.com. Conde Nast Digital, 09 Nov. 0014. Web. 01 Dec. 2014.

\bibitem{humanity}Dillow, Clay. "Will People Alive Today Have the Opportunity to
	Upload Their Consciousness to a New Robotic Body?" Popular Science. Popular
	Science, 02 Mar. 2012. Web. 01 Dec. 2014.

\bibitem{Theseus}Chisholm, Roderick Milton. Person and Object: A Metaphysical
	Study. London: Routledge, 2002. Print.

\bibitem{Kurzweil}Kurzweil, Ray. "The Singularity Is Near: When Humans Transcend
	Biology." - AbeBooks. N.p., n.d. Web. 01 Dec. 2014.

\bibitem{itu}"ITU: Committed to Connecting the World." ITU. N.p., n.d. Web. 02
	Dec. 2014.

\end{thebibliography}

\end{document}
