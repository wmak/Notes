\documentclass[12pt]{article}

\usepackage[margin=1in]{geometry}
\usepackage{hyperref}

\title{Chinese Representation in Media}
\author{William Mak - 998992988}
\date{June, 24, 2014}

\begin{document}

\maketitle

	Various cutural groups have a wide range of representations in media, their
	representations cause associations or stereotypes to be made with their
	respective groups. Speicifically within the representation of asians within
	media, I've seen that they are often a minority within non-asian media. An
	example of this would be from the TV series Lost which finished airing
	several years ago. Within this show their were a group of airline passengers
	who become stranded on an island. Within this group there were only two
	Asian passengers, which defintely made them the minority of the group. While
	in another TV series `Gilded Chopsticks` a Hong Kong TV drama set in ancient
	China there were no other cultural group except the Chinese. It is then in
	my opinion that from my experience with media Asians are neither over or
	under represented. In the context of Lost, the flight was from Australia a
	country whose population is not majorly Asian, the survivors of the crash
	were also tourists which made sense for the show. If instead a majority of
	the passengers were Asian then it would seem that they were over
	represented, or if there weren't any at all then it would appear that they
	were over represented. While in the other example; being in ancient China
	it would be illogical if Asians were a minority. Therefore then Asians are
	neither over or under represented in media, they are represented in a
	logical and adequate way.
	
	It's also known that with each cultural group there comes with it various
	associations, and these associations come with there own connotations some 
	negative for example there is an entire comedy series on youtube devoted to
	'educating' asians on how to drive. The video continues to show several
	other stereotypes, squinted eyes, bowing, having smaller genitilia, being
	hard to tell apart	and clasping ones hands as a
	greeting. While this video is very offensive 
	these associations are negative but don't show up as often in mass
	media and is more likely to be found in comedy. Of course there are also 
	associations that are positive, for
	example in the Rush Hour series Jackie Chan plays a character that fills
	several stereotypes of Asians; being good at kung fu, drives a taxi, soft
	spoken, being good at math and overworking. These traits are generally
	positive but can still be viewed as negative depending on ones point of view 
	Overall it can be seen that 
	Asian representation in media is not over or under represented. Generally 
	the media shows a decent balance of positive and negative stereotypes. I
	think media in general is also quite aware of the stereotypes they make as
	seen in this \href{https://www.youtube.com/watch?v=VDjgfWAL5UQ}{Video}
\end{document}

